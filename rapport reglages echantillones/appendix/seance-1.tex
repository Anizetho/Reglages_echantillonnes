\chapter{Séance 1: Codes matlab}\label{annexe-1}
La fonction ci-dessous permet de discrétiser un système d'après les méthodes des différences finies (\textit{gauche, droite centrale)} et par l'équivalent échantilloné bloqué. \\
Les paramètres utilisés sont:
\begin{itemize}[label=$\cdot$]
\item H:  expression (en $s$) du système
\item pe: la période d'échantillonnage.
\end{itemize} 
\paragraph{}
Les résultats obtenus sont ensuite comparés sous leur formes temporelles et fréquentielles.

{\footnotesize\begin{verbatim}
%function s1_plot(H,pe)
z   = tf('z',pe);

% équivalent bloqué & signal de base
Hb  = c2d(H,pe);
[yb,t] = step(Hb);
ye  = 1-exp(-t);

%Différences finies à gauche
sg = (z-1)/pe;
Hg = 1/(sg+1);
yg = step(Hg);

%Différences finies à droite
sd = (z-1)/(pe*z);
Hd = 1/(sd+1);
yd = step(Hd);

%Différences finies central
sc = (z-1)/(z+1)*2/pe;
Hc = 1/(sc+1);
yc = step(Hc);

[~,~,wout]    = bode(H);






% PLOT
figure

subplot(1,3,1)
plot(t,ye,'-*',t,yg,'-*',t,yd,'-*',t,yc,'-*',t,yb,'-')
title(sprintf('réponse temporelle pour une pe = %f',pe))
legend('Hb','Hg','Hd','Hc','H','Location','southeast');grid
subplot(1,3,2)
title('erreur relative')
plot(t,yb-ye,'-*',t,yg-ye,'-*',t,yd-ye,'-*',t,yc-ye,'-*')
title('erreur relative')
legend('Hb','Hg','Hd','Hc','Location','southeast');grid
subplot(1,3,3)
bode(Hb,Hg,Hd,Hc,H,wout)
legend('Hb','Hg','Hd','Hc','H','Location','southwest');grid
\end{verbatim}}

\paragraph{}
La fonction précédente a été utilisée dans l'exercice 2 pour le système de premier ordre $H(s)=\frac{1}{s+1}$
{\footnotesize\begin{verbatim}
s   = tf('s');
H   = 1/(s+1);
pe  = 1; 
\end{verbatim}}

\paragraph{}
Ainsi qu'au point 4 pour un troisième ordre:
{\footnotesize\begin{verbatim}
H2  = 1/(s+1)^3;
H2b = c2d(H2,pe);
\end{verbatim}}